\documentclass[journal,12pt,twocolumn]{IEEEtran}

\usepackage{setspace}
\usepackage{gensymb}

\singlespacing


\usepackage[cmex10]{amsmath}

\usepackage{amsthm}

\usepackage{mathrsfs}
\usepackage{txfonts}
\usepackage{stfloats}
\usepackage{bm}
\usepackage{cite}
\usepackage{cases}
\usepackage{subfig}

\usepackage{longtable}
\usepackage{multirow}

\usepackage{enumitem}
\usepackage{mathtools}
\usepackage{steinmetz}
\usepackage{tikz}
\usepackage{circuitikz}
\usepackage{verbatim}
\usepackage{tfrupee}
\usepackage[breaklinks=true]{hyperref}
\usepackage{graphicx}
\usepackage{tkz-euclide}

\usetikzlibrary{calc,math}
\usepackage{listings}
    \usepackage{color}                                            %%
    \usepackage{array}                                            %%
    \usepackage{longtable}                                        %%
    \usepackage{calc}                                             %%
    \usepackage{multirow}                                         %%
    \usepackage{hhline}                                           %%
    \usepackage{ifthen}                                           %%
    \usepackage{lscape}     
\usepackage{multicol}
\usepackage{chngcntr}

\DeclareMathOperator*{\Res}{Res}

\renewcommand\thesection{\arabic{section}}
\renewcommand\thesubsection{\thesection.\arabic{subsection}}
\renewcommand\thesubsubsection{\thesubsection.\arabic{subsubsection}}

\renewcommand\thesectiondis{\arabic{section}}
\renewcommand\thesubsectiondis{\thesectiondis.\arabic{subsection}}
\renewcommand\thesubsubsectiondis{\thesubsectiondis.\arabic{subsubsection}}


\hyphenation{op-tical net-works semi-conduc-tor}
\def\inputGnumericTable{}                                 %%

\lstset{
%language=C,
frame=single, 
breaklines=true,
columns=fullflexible
}
\begin{document}


\newtheorem{theorem}{Theorem}[section]
\newtheorem{problem}{Problem}
\newtheorem{proposition}{Proposition}[section]
\newtheorem{lemma}{Lemma}[section]
\newtheorem{corollary}[theorem]{Corollary}
\newtheorem{example}{Example}[section]
\newtheorem{definition}[problem]{Definition}

\newcommand{\BEQA}{\begin{eqnarray}}
\newcommand{\EEQA}{\end{eqnarray}}
\newcommand{\define}{\stackrel{\triangle}{=}}
\bibliographystyle{IEEEtran}
\providecommand{\mbf}{\mathbf}
\providecommand{\pr}[1]{\ensuremath{\Pr\left(#1\right)}}
\providecommand{\qfunc}[1]{\ensuremath{Q\left(#1\right)}}
\providecommand{\sbrak}[1]{\ensuremath{{}\left[#1\right]}}
\providecommand{\lsbrak}[1]{\ensuremath{{}\left[#1\right.}}
\providecommand{\rsbrak}[1]{\ensuremath{{}\left.#1\right]}}
\providecommand{\brak}[1]{\ensuremath{\left(#1\right)}}
\providecommand{\lbrak}[1]{\ensuremath{\left(#1\right.}}
\providecommand{\rbrak}[1]{\ensuremath{\left.#1\right)}}
\providecommand{\cbrak}[1]{\ensuremath{\left\{#1\right\}}}
\providecommand{\lcbrak}[1]{\ensuremath{\left\{#1\right.}}
\providecommand{\rcbrak}[1]{\ensuremath{\left.#1\right\}}}
\theoremstyle{remark}
\newtheorem{rem}{Remark}
\newcommand{\sgn}{\mathop{\mathrm{sgn}}}
\providecommand{\abs}[1]{\left\vert#1\right\vert}
\providecommand{\res}[1]{\Res\displaylimits_{#1}} 
\providecommand{\norm}[1]{\left\lVert#1\right\rVert}
%\providecommand{\norm}[1]{\lVert#1\rVert}
\providecommand{\mtx}[1]{\mathbf{#1}}
\providecommand{\mean}[1]{E\left[ #1 \right]}
\providecommand{\fourier}{\overset{\mathcal{F}}{ \rightleftharpoons}}
%\providecommand{\hilbert}{\overset{\mathcal{H}}{ \rightleftharpoons}}
\providecommand{\system}{\overset{\mathcal{H}}{ \longleftrightarrow}}
	%\newcommand{\solution}[2]{\textbf{Solution:}{#1}}
\newcommand{\solution}{\noindent \textbf{Solution: }}
\newcommand{\cosec}{\,\text{cosec}\,}
\providecommand{\dec}[2]{\ensuremath{\overset{#1}{\underset{#2}{\gtrless}}}}
\newcommand{\myvec}[1]{\ensuremath{\begin{pmatrix}#1\end{pmatrix}}}
\newcommand{\mydet}[1]{\ensuremath{\begin{vmatrix}#1\end{vmatrix}}}
\numberwithin{equation}{subsection}
\makeatletter
\@addtoreset{figure}{problem}
\makeatother
\let\StandardTheFigure\thefigure
\let\vec\mathbf
\renewcommand{\thefigure}{\theproblem}
\def\putbox#1#2#3{\makebox[0in][l]{\makebox[#1][l]{}\raisebox{\baselineskip}[0in][0in]{\raisebox{#2}[0in][0in]{#3}}}}
     \def\rightbox#1{\makebox[0in][r]{#1}}
     \def\centbox#1{\makebox[0in]{#1}}
     \def\topbox#1{\raisebox{-\baselineskip}[0in][0in]{#1}}
     \def\midbox#1{\raisebox{-0.5\baselineskip}[0in][0in]{#1}}
\vspace{3cm}
\title{Assignment-9}
\author{Ankur Aditya - EE20RESCH11010}
\maketitle
\newpage
\bigskip
\renewcommand{\thefigure}{\theenumi}
\renewcommand{\thetable}{\theenumi}

\begin{abstract}
This document contains the procedure to find the row reduced matrix of a given 3$\times$3 matrix. 
\end{abstract}
Download the python code from 
\begin{lstlisting}
https://github.com/ankuraditya13/EE5609-Assignment9
\end{lstlisting}
%
and latex-file codes from 
%
\begin{lstlisting}
https://github.com/ankuraditya13/EE5609-Assignment9
\end{lstlisting}

\section{Problem}
Find a row-reduced matrix which is row equivalent to,
\begin{align}
\vec{A} = \myvec{i&-(1+i)&0\\1&-2&1\\1&2i&-1}
\label{A}
\end{align}
\section{Solution}
\textbf{Step 1}: Performing scaling operation to matrix $\vec{A}$ as $R_1\leftarrow \frac{1}{i}R_1$ by scaling matrix $D_1$ given as,
\begin{align}
\vec{D_1} = \myvec{\frac{1}{i}&0&0\\0&1&0\\0&0&1}\\
\vec{D_1A} = \myvec{\frac{1}{i}&0&0\\0&1&0\\0&0&1}\myvec{i&-(1+i)&0\\1&-2&1\\1&2i&-1}\\
\implies\vec{D_1A} = \myvec{1&-1+i&0\\1&-2&1\\1&2i&-1}\label{1}
\end{align}
\textbf{Step 2}: Performing $R_2 \leftarrow R_2-R_1$ and $R_3 \leftarrow R_3-R_1$ given by elementary matrix $\vec{E_{31}E_{21}}$ on equation \eqref{1},
\begin{align}
\vec{E_{31}E_{21}} = \myvec{1&0&0\\-1&1&0\\-1&0&1}\\
\vec{E_{31}E_{21}D_1A} = \myvec{1&0&0\\-1&1&0\\-1&0&1}\myvec{1&-1+i&0\\1&-2&1\\1&2i&-1}\\
\implies\vec{A_1} = \vec{E_{31}E_{21}D_1A} = \myvec{1&-1+i&0\\0&-1-i&1\\0&1+i&-1}\label{2}
\end{align}
\textbf{Step 3}: Performing $R_2\leftarrow \frac{-1}{1+i}R_2$ given by $\vec{D_2}$ on equation \eqref{2},
\begin{align}
\vec{D_2} = \myvec{1&0&0\\0&\frac{1}{2}(-1+i)&0\\0&0&1}\\
\vec{D_2A_1} = \myvec{1&0&0\\0&\frac{1}{2}(-1+i)&0\\0&0&1}\myvec{1&-1+i&0\\0&-1-i&1\\0&1+i&-1}\\
\implies \vec{A_2}=\vec{D_2A_1} = \myvec{1&-1+i&0\\0&1&\frac{1}{2}(-1+i)\\0&1+i&-1}\label{3}
\end{align}
\textbf{Step 4}: Performing $R_3\leftarrow R_3-(1+i)R_2$ given by $\vec{E_{32}}$ on equation \eqref{3},
\begin{align}
\vec{E_{32}} = \myvec{1&0&0\\0&1&0\\0&-(1+i)&1}
\end{align}
\begin{align}
\vec{E_{32}A_2} = \myvec{1&0&0\\0&1&0\\0&-1-i&1}\myvec{1&-1+i&0\\0&1&\frac{-1+i}{2}\\0&1+i&-1}
\end{align}
\begin{align}
\implies \vec{A_3}=\vec{E_{32}A_2} = \myvec{1&-1+i&0\\0&1&\frac{-1+i}{2}\\0&0&1}\label{4}
\end{align}
\textbf{Step 5}: Performing $R_1\leftarrow R_1-(-1+i)R_2$ given by $\vec{E_{12}}$ on equation \eqref{4},
\begin{align}
\vec{E_{12}} = \myvec{1&1-i&0\\0&1&0\\0&0&1}
\end{align}
\begin{align}
\vec{E_{12}A_3} = \myvec{1&1-i&0\\0&1&0\\0&0&1}\myvec{1&-1+i&0\\0&1&\frac{-1+i}{2}\\0&0&1}
\end{align}
\begin{align}
\implies \vec{A_4} = \vec{E_{12}A_3} = \myvec{1&0&i\\0&1&\frac{-1+i}{2}\\0&0&1}\label{5}
\end{align}
\textbf{Step 6}: Performing $R_1\leftarrow R_1-iR_3$ and $R_2\leftarrow R_2-\frac{-1+i}{2}R_3$ given by $\vec{E_{13}E_{23}}$ on equation \eqref{5},
\begin{align}
\vec{E_{13}E_{23}} = \myvec{1&0&-i\\0&1&-\brak{\frac{-1+i}{2}}\\0&0&1}
\end{align}
\begin{align}
\vec{E_{13}E_{23}A_4} = \myvec{1&0&-i\\0&1&-\brak{\frac{-1+i}{2}}\\0&0&1}\myvec{1&0&i\\0&1&\frac{-1+i}{2}\\0&0&1}\\
\implies \vec{A_5} = \vec{E_{13}E_{23}A_4} = \myvec{1&0&0\\0&1&0\\0&0&1} 
\end{align}
$\therefore$ Row-reduced matrix of $\vec{A}$ given by equation \eqref{A} is,
\begin{align}
\vec{A} = \myvec{i&-1-i&0\\1&-2&1\\1&2i&-1}\xleftrightarrow[]{RREF}\myvec{1&0&0\\0&1&0\\0&0&1} = \vec{I}
\end{align}
\end{document}